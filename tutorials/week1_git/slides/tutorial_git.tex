% !TEX encoding = UTF-8 Unicode
\documentclass{beamer}

\usepackage{amsmath}
\usepackage{color}
\usepackage{gensymb}
\usepackage{hyperref}
\usepackage{textcomp}
\usepackage{tikz}
\usepackage{verbatim}

\usetheme{Warsaw}

\definecolor{darkolivegreen}{rgb}{0.33, 0.42, 0.18}

\title[ESS 490-590 - Git tutorial]{Introduction to version control with git}
\author{Ariane Ducellier}
\institute{University of Washington}
\date{ESS 490-590 Data science for Earth and planetary systems - Spring 2021}

\begin{document}

	\begin{frame}
		\titlepage
	\end{frame}

	\section*{Anaconda}

	\begin{frame}
	\frametitle{What do you need to run your Python code?}
	\begin{itemize}
	\setlength\itemsep{1em}
		\item Your code: Some files with .py extension
		\item Python package: Libraries that contains many functions related to each other (e.g. numpy, scipy, pandas, scikit-learn)
		\item To run your code, you need to define its environment: Version of Python + Some packages + Version of packages
	\end{itemize}
	\end{frame}

	\begin{frame}
	\frametitle{When do you need to know your environment?}
	\begin{itemize}
	\setlength\itemsep{1em}
		\item You may have on your computer different Python codes with different versions of packages
		\item You give your code to a friend
		\item Some of your packages may depend on other packages, with a specific version. How to make sure you have the right version of everything?
	\end{itemize}
	\end{frame}

	\begin{frame}
	\frametitle{How to deal with this?}
	Install anaconda: \href{https://www.anaconda.com/products/individual}{https://www.anaconda.com/products/individual}

	\begin{itemize}
	\setlength\itemsep{1em}
		\item User interface + command line
		\item Tools for developing code in Python: JupyterLab, Spyder
		\item  Jupyter notebook (more on this later)
		\item Tools for managing environment
	\end{itemize}
	\end{frame}

	\begin{frame}[fragile]
	\frametitle{Basic conda commands}
	Check conda version to make sure it's installed
	\begin{exampleblock}{}
		\begin{verbatim}
		conda info
		\end{verbatim}
	\end{exampleblock}

	\vspace{0.5cm}

	List out available environments (the starred * environment is the current activate environment)
	\begin{exampleblock}{}
		\begin{verbatim}
		conda env list 
		\end{verbatim}
	\end{exampleblock}
	\end{frame}

	\begin{frame}[fragile]
	\frametitle{Basic conda commands (continued)}
	Create conda environment from environment file
	\begin{exampleblock}{}
		\begin{verbatim}
		conda env create --file environment.yml
		\end{verbatim}
	\end{exampleblock}

	\vspace{0.5cm}

	Removing conda environment
	\begin{exampleblock}{}
		\begin{verbatim}
		conda env remove --yes --name myenv
		\end{verbatim}
	\end{exampleblock}
	\end{frame}

	\begin{frame}[fragile]
	\frametitle{Basic conda commands (continued)}
	Activate conda environment
	\begin{exampleblock}{}
		\begin{verbatim}
		conda activate myenv
		\end{verbatim}
	\end{exampleblock}

	\vspace{0.5cm}

	Deactivate conda environment
	\begin{exampleblock}{}
		\begin{verbatim}
		conda deactivate
		\end{verbatim}
	\end{exampleblock}
	\end{frame}

	\begin{frame}[fragile]
	\frametitle{Example of .yml file}
	\begin{exampleblock}{}
		\begin{verbatim}
		name: MLlabs
		channels:
		  - conda-forge
		  - defaults
		dependencies:
		  - python=3.9
		  - jupyter
		  - matplotlib
		  - numpy
		  - pandas
		  - scipy
		  - scikit-learn
		  - pytorch
		\end{verbatim}
	\end{exampleblock}
	\end{frame}

	\begin{frame}
	\frametitle{To learn more about Anaconda environments}
        	\href{https://docs.conda.io/projects/conda/en/latest/user-guide/tasks/manage-environments.html}{https://docs.conda.io/projects/conda/en/latest/user-guide/tasks/manage-environments.html}
	\end{frame}

	\section*{Jupyter notebooks}

    	\begin{frame}
	\frametitle{How to explain what your code is doing?}
	A Jupyter notebooks allows you to merge:
	\begin{itemize}
	\setlength\itemsep{1em}
		\item Text
		\item Images
		\item Code
		\item Output of your code
	\end{itemize}
	\end{frame}

	\begin{frame}
	\frametitle{What is a Jupyter notebook made of?}
	\begin{itemize}
	\setlength\itemsep{1em}
		\item Markdown cells for the text
		\item Code cells for the code
		\item Kernel:
		\begin{itemize}
			\item You can run the code cells one by one
			\item Run all the cells until the end
			\item Restart the kernel i.e. delete all the variables that you've created so far
		\end{itemize}
	\end{itemize}
	\end{frame}

	\section*{Markdown}

	\begin{frame}
	\frametitle{What is Markdown?}
	Simple language to format text

	Used for:
	\begin{itemize}
		\item Text in Jupyter notebooks
		\item Text on .md files on GitHub (e.g. README.md in a GitHub repository)
		\item Text on RStudio files
	\end{itemize}
	\end{frame}

	\begin{frame}[fragile]
	\frametitle{Basic Markdown commands}
	Headings
	\begin{exampleblock}{}
		\begin{verbatim}
		# Heading level 1
		## Heading level 2
		### Heading level 3
		\end{verbatim}
	\end{exampleblock}
	\end{frame}

	\begin{frame}[fragile]
	\frametitle{Basic Markdown commands (continued)}
	Paragraphs: Leave a blank line
	\begin{exampleblock}{}
		\begin{verbatim}
		This is my first paragraph.

		This is my second paragraph.
		\end{verbatim}
	\end{exampleblock}

	\vspace{0.5cm}

	Line break: Leave two or more spaces
	\begin{exampleblock}{}
		\begin{verbatim}
		This is my first line.   
		This is my second line.
		\end{verbatim}
	\end{exampleblock}
	\end{frame}

	\begin{frame}[fragile]
	\frametitle{Basic Markdown commands (continued)}
	Bold text
	\begin{exampleblock}{}
		\begin{verbatim}
		**This is my bold text**
		\end{verbatim}
	\end{exampleblock}

	\vspace{0.5cm}

	Italic text
	\begin{exampleblock}{}
		\begin{verbatim}
		*This is my italic text*
		\end{verbatim}
	\end{exampleblock}

	\vspace{0.5cm}

	Bold and italic text
	\begin{exampleblock}{}
		\begin{verbatim}
		***This is my bold and italic text***
		\end{verbatim}
	\end{exampleblock}
	\end{frame}

	\begin{frame}[fragile]
	\frametitle{Basic Markdown commands (continued)}
	Ordered list
	\begin{exampleblock}{}
		\begin{verbatim}
		1. First item
		8. Second item
		3. Third item
		5. Fourth item
		\end{verbatim}
	\end{exampleblock}

	\vspace{0.5cm}

	Bullet points
	\begin{exampleblock}{}
		\begin{verbatim}
		- First item
		- Second item
		- Third item
		- Fourth item
		\end{verbatim}
	\end{exampleblock}
	\end{frame}

	\begin{frame}[fragile]
	\frametitle{Basic Markdown commands (continued)}
	Images
	\begin{exampleblock}{}
		\begin{verbatim}
		<img src="images/glass.png" width="200"/>
		\end{verbatim}
	\end{exampleblock}

	\vspace{0.5cm}

	Equations (\LaTeX style)
	\begin{exampleblock}{}
		\begin{verbatim}
		$\frac{x}{y} = \sqrt{z}$
		\end{verbatim}
	\end{exampleblock}
	\end{frame}

	\begin{frame}
	\frametitle{To learn more about Markdown}
        	\href{https://www.markdownguide.org/}{https://www.markdownguide.org/}
	\end{frame}

	\section*{Git}

	\begin{frame}
	\frametitle{First step: Create a GitHub account}
	Go to \href{https://github.com/}{https://github.com/} and sign up on the top right corner of the webpage.
	\end{frame}

	\begin{frame}
	\frametitle{Second step: Install git on your computer}
	Download git from \href{https://git-scm.com/downloads}{https://git-scm.com/downloads}
	\end{frame}

	\begin{frame}
	\frametitle{Why would you want to use git?}
	\begin{itemize}
	\setlength\itemsep{1em}
		\item Version control: Keep track of the changes (get back to the version of your code which was working)
		\item Let two or more people work on the same file, and keep track of all modifications
	\end{itemize}
	\end{frame}

	\begin{frame}
	\frametitle{What is the difference between git and GitHub?}
	\begin{itemize}
	\setlength\itemsep{1em}
		\item git is a software: You can use it to track changes only on your computer
		\item GitHub is a platform that can host your repository. There are other platforms like Bitbucket.
	\end{itemize}
	\end{frame}

	\begin{frame}[fragile]
	\frametitle{If you have never used git on your computer}
	You need to set up your user name and your e-mail
	\begin{exampleblock}{}
		\begin{verbatim}
		> git config --global user.name "ArianeDucellier"
		> git config --global user.email "ducela@uw.edu"
			\end{verbatim}
	\end{exampleblock}
	Use the same user name as your GitHub account. On GitHub, you will see who has made the modifications to the repository.

	You need to do it only once on your computer.
	\end{frame}

	\begin{frame}
	\frametitle{Let us create a new repository on GitHub}
	\begin{itemize}
	\setlength\itemsep{1em}
		\item Connect to your GitHub account
		\item On the top right of your homepage, click "+" and New repository
		\item Check "Add a README.md"
		\item Create repository
	\end{itemize}
	\end{frame}

	\begin{frame}
	\frametitle{Let us do our first commit}
	\begin{itemize}
	\setlength\itemsep{1em}
		\item You can edit the README.md file by going on the pen icon
		\item Click commit to keep track of the modification
		\item You can now see that there are now two commits to your repository: Initial commit and Update README.md
	\end{itemize}
	\end{frame}

	\begin{frame}[fragile]
	\frametitle{Let us now work on our computer}
	The repository now exists on GitHub but not on your computer.
	\begin{exampleblock}{}
		\scriptsize{
		\begin{verbatim}
		> git clone "https://github.com/ArianeDucellier/example.git"
		\end{verbatim}
		}
	\end{exampleblock}

	\vspace{0.5cm}

	\begin{exampleblock}{}
		\begin{verbatim}
		> cd example
		> ls -al
		> cat README.md
		\end{verbatim}
	\end{exampleblock}
	\end{frame}

	\begin{frame}[fragile]
	\frametitle{Let us create a file in our repository}
	\begin{exampleblock}{}
		\begin{verbatim}
		> git status
		> touch my_file.txt
		> nano my_file.txt
		> git status
		\end{verbatim}
	\end{exampleblock}
	\end{frame}

	\begin{frame}[fragile]
	\frametitle{Let us add the file to the next commit}
	\begin{exampleblock}{}
		\begin{verbatim}
		> git add my_file.txt 
		> git commit -m "Created file"
		> git status
		\end{verbatim}
	\end{exampleblock}
	My branch is ahead of 'origin/main' by 1 commit.
	\end{frame}

	\begin{frame}[fragile]
	\frametitle{Let us upload the modifications on GitHub}
	After making a commit
	\begin{exampleblock}{}
		\begin{verbatim}
		> git push
		\end{verbatim}
	\end{exampleblock}

	\vspace{0.5cm}

	Always before starting working in your repository
	\begin{exampleblock}{}
		\begin{verbatim}
		> git pull
		\end{verbatim}
	\end{exampleblock}
	\end{frame}

	\begin{frame}[fragile]
	\frametitle{Let us look at the history of our repository}
	\begin{itemize}
	\setlength\itemsep{1em}
		\item Update my\_file.txt, commit and push
		\item On GitHub, click on my\_file.txt and look at history
		\item On your computer, with the command line:
	\end{itemize}
	\begin{exampleblock}{}
		\begin{verbatim}
		> git log
		> git log --oneline
		\end{verbatim}
	\end{exampleblock}
	\end{frame}

	\begin{frame}[fragile]
	\frametitle{Come back to a previous version of our file}
	Make an additional modification to my\_file.txt

	\vspace{0.5cm}

	\begin{exampleblock}{}
		\begin{verbatim}
		> git checkout my_file.txt
		> cat my_file.txt
		> git checkout fa708e8 my_file.txt
		> cat my_file.txt
		\end{verbatim}
	\end{exampleblock}
	\end{frame}

	\begin{frame}[fragile]
	\frametitle{How to not keep track of some files}
	\begin{exampleblock}{}
		\begin{verbatim}
		> touch data.dat
		> git status
		> touch .gitignore
		> nano .gitignore
		> git status
		> git status --ignored
		> git add -f data.dat
		\end{verbatim}
	\end{exampleblock}
	\end{frame}

	\begin{frame}
	\frametitle{Work with other people}
	In your repository on GitHub:

	\vspace{0.5cm}

	\begin{itemize}
	\setlength\itemsep{1em}
		\item Go to Settings
		\item Go to Manage access
		\item Go to Invite a collaborator
	\end{itemize}
	\end{frame}

	\begin{frame}
	\frametitle{Dealing with conflicts between two versions}
	Create a conflict:
	\begin{itemize}
		\item Modify my\_file.txt on GitHub and commit on GitHub
		\item Commit the last modifications to my\_file.txt on your computer and push
	\end{itemize}
	$\rightarrow$ You get an error message
	\begin{itemize}
		\item Pull the last version on GitHub
		\item Open my\_file.txt and resolve the conflict
		\item Add my\_file.txt, commit, push
		\item Check the history of your commits
	\end{itemize}
	\end{frame}

	\begin{frame}
	\frametitle{To learn more about git}
        	\href{http://swcarpentry.github.io/git-novice/}{http://swcarpentry.github.io/git-novice/}
	\end{frame}


	\begin{frame}
	\frametitle{Bonus: Binder}
	
	\href{https://docs.google.com/presentation/d/1IP_0RclPDHT3ezbdFjxc8D0xwMJIhWNnnNgTWkbKyVI/edit?usp=sharing}{Binder presentation}

	\vspace{0.5cm}

	\href{https://github.com/choldgraf/uw-reproducibility-2019}{Binder workshop}

	\vspace{0.5cm}

	\href{https://the-turing-way.netlify.com/introduction/introduction}{The Turing way}

	\end{frame}

	\begin{frame}
	\centering
		\Huge{Happy code sharing!}
	\end{frame}

\end{document}
